\section*{Question 1}

\subsection*{a.}

\noindent
$\ket{\psi}$ is already normalized:

\begin{flalign*}
    \braket{\psi | \psi} &= \left( \frac{1}{\sqrt{2}}\bra{\phi_1} - \frac{i}{\sqrt{2}}\bra{\phi_2} \right) \left( \frac{1}{\sqrt{2}}\ket{\phi_1} + \frac{i}{\sqrt{2}}\ket{\phi_2} \right) \\\\
                         &= \frac{1}{2}\braket{\phi_1 | \phi_1} + \frac{i}{2}\braket{\phi_2 | \phi_1} - \frac{i}{2}\braket{\phi_1 | \phi_2} + \frac{1}{2}\braket{\phi_2 | \phi_2} \\
\end{flalign*}

\noindent
Since $\ket{\phi_1}$ and $\ket{\phi_2}$ are eigenstates of a Hermitian operator, they are orthogonal to each other. \\

\noindent
Thus, $\braket{\phi_1 | \phi_2} = \braket{\phi_2 | \phi_1} = 0$, and $\braket{\phi_1 | \phi_1} = \braket{\phi_2 | \phi_2} = 1$, meaning:

\begin{flalign*}
    & \braket{\psi | \psi} = 1
\end{flalign*}

\subsection*{b.}

\noindent
A measurement made on $\ket{\psi}$ with respect to $\matbf{\Phi}$ can only yield either $\phi_1$ or $\phi_2$,
since they are the eigenvales of $\matbf{\Phi}$. \\

\noindent
The probability of measuring $\phi_1$ is given by $P(\phi_1) = \left| \braket{\phi_1 | \psi} \right|^2$.

\begin{flalign*}
    \braket{\phi_1 | \psi} &= \bra{\phi_1} \left( \frac{1}{\sqrt{2}}\ket{\phi_1} + \frac{i}{\sqrt{2}}\ket{\phi_2} \right) \\
                           &= \frac{1}{\sqrt{2}}\braket{\phi_1 | \phi_1} + \frac{i}{\sqrt{2}}\braket{\phi_1 | \phi_2} \\
                           &= \frac{1}{\sqrt{2}}
\end{flalign*}

\noindent
Thus,

\begin{flalign*}    
    & P(\phi_1) = \left| \frac{1}{\sqrt{2}} \right|^2 = \frac{1}{2} \\
\end{flalign*}

\noindent
Similarly, for $\phi_2$, $P(\phi_2) = \left| \braket{\phi_2 | \psi} \right|^2$.

\begin{flalign*}
    \braket{\phi_2 | \psi} &= \bra{\phi_2} \left( \frac{1}{\sqrt{2}}\ket{\phi_1} + \frac{i}{\sqrt{2}}\ket{\phi_2} \right) \\
                           &= \frac{1}{\sqrt{2}}\braket{\phi_2 | \phi_1} + \frac{i}{\sqrt{2}}\braket{\phi_2 | \phi_2} \\
                           &= \frac{i}{\sqrt{2}}
\end{flalign*}

\noindent
Thus,

\begin{flalign*}    
    & P(\phi_2) = \left| \frac{i}{\sqrt{2}} \right|^2 = \frac{1}{2}
\end{flalign*}

\subsection*{c.}

% I need to find another quantum state that is orthogonal to Psi

\noindent
To find another quantum state that is orthogonal to $\ket{\Psi}$, we can set the inner product of the two states to 0. \\

\noindent
Let $\ket{\Psi_2} = \alpha\ket{\phi_1} + \beta\ket{\phi_2}, \; \phi_1, \phi_2 \in \mathbb{C}$,

\begin{flalign*}
    \braket{\Psi | \Psi_2} = 0 &= \left( \frac{1}{\sqrt{2}}\bra{\phi_1} - \frac{i}{\sqrt{2}}\bra{\phi_2} \right) \left( \alpha\ket{\phi_1} + \beta\ket{\phi_2} \right) \\
                               &= \frac{\alpha}{\sqrt{2}}\braket{\phi_1 | \phi_1} + \frac{\beta}{\sqrt{2}}\braket{\phi_2 | \phi_1} - \frac{i\alpha}{\sqrt{2}}\braket{\phi_1 | \phi_2} - \frac{i\beta}{\sqrt{2}}\braket{\phi_2 | \phi_2} \\
                               &= \frac{\alpha}{\sqrt{2}} - \frac{i\beta}{\sqrt{2}} \\
                               &= \frac{1}{\sqrt{2}}(\alpha - i\beta) \\
\end{flalign*}

\noindent
Thus, we get that $0 = \alpha - i\beta$. Let $\alpha = x_{\alpha} + iy_{\alpha}$ and $\beta = x_{\beta} + iy_{\beta}$, where $x_{\alpha}, y_{\alpha}, x_{\beta}, y_{\beta} \in \mathbb{R}$. \\

\noindent
Then,

\begin{flalign*}
    0 &= \alpha - i\beta \\
    0 &= (x_{\alpha} + iy_{\alpha}) - i(x_{\beta} + iy_{\beta}) \\
    0 &= x_{\alpha} + iy_{\alpha} - ix_{\beta} + y_{\beta} \\
    0 &= (x_{\alpha} + y_{\beta}) + i(y_{\alpha} - x_{\beta}) \\
\end{flalign*}

\noindent
So, $x_{\alpha} = -y_{\beta}$ and $x_{\beta} = y_{\alpha}$. Let $\alpha = \frac{1}{\sqrt{2}}(1 + i)$ and $\beta = \frac{1}{\sqrt{2}}(1 - i)$. \\

\noindent
Thus, another quantum state that is orthogonal to $\ket{\Psi}$ is:

\begin{flalign*}
    & \ket{\Psi_2} = \frac{1}{\sqrt{2}}(1 + i)\ket{\phi_1} + \frac{1}{\sqrt{2}}(1 - i)\ket{\phi_2}
\end{flalign*}

\subsection*{d.}

% Calculate the probability of finding the system in the state Psi_2 if a measurement is made with respect to Phi

\noindent
The probability of finding the system in the state $\ket{\Psi_2}$ if a if a measurement is made is given by
$P(\Psi_2) = \left| \braket{\Psi_2 | \Psi} \right|^2$. \\

\noindent
Since $\ket{\Psi_2}$ is orthogonal to $\ket{\Psi}$, $\braket{\Psi_2 | \Psi} = 0$, meaning that
the probability of finding the system in the state $\ket{\Psi_2}$ is 0.